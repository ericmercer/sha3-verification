\begin{abstract}
The Keccak function in the SHA-3 NIST standard is responsible for the major computations of all SHA-3 hashes and Shake functions.
These functions are a key piece in modern secure hash functions. 
This work extends and improves on the methodology used in the verification of Amazon’s s2n HMAC and OpenSSL’s SHA2-256 inner hash function. 
It uses the Software Analysis Workbench (SAW) to prove out functional equivalence between OpenSSL’s SHA-3 Keccak function and a verified implementation of SHA-3 in the domain-specific language Cryptol. 
SAW creates contracts (called overrides), a functions input to output relationship, which allows for a modular approach to writing proofs. 
This work leverages SAW by creating overrides to block out computation and creating a compositional proof that remains within the capacity of an SMT solver. 
This work amends Cryptol through utility functions to handle looping which greatly simplifies the visual inspection the Cryptol implementation to provide strong assurance of correctness. 
The results of this work indicate that this methodology works with modern cryptographic specifications and that a modular approach in cryptographic algorithm design greatly simplifies the verification process.
\end{abstract}