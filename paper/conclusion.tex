This paper presents the proof of the \shaThree\ implementation in \openssl\ using the \saw\ tool.
The proof is accomplished by writing a \cryptol\ specification for the \nist\ \fips\ specification that defines the \shaThree\ algorithm.
The \cryptol\ specification is aided with two additional \cryptol\ library functions, the for-method and while-method, to hide the functional list comprehensions.
The for-method is a parallel-for operating on each element in a list, and the while-method is a list iteration.
The \cryptol\ is shown equivalent to the \fips\ standard through visual inspection and input to output matching on published test vectors.

The proof uses a second \cryptol\ specification based on the actual \openssl\ C implementation of \shaThree.
This specification reorders arrays in the specification to match the C standard following row major order.
It also includes table lookups for constant values in the computation but it computes those tables to show the provenance of the constants.
That provenance is needed later for the proof of equivalence with the C code.
\saw\ proves the input to output relationship equivalent between the \fips\ specification and this revised specification for the 256-bit digest on any message of size $0$ to $1,088$ bits.

The final \saw\ proof shows the \keccak\ function in the revised specification is equivalent to the \keccak\ function in the actual C implementation.
\keccak\ is the core of the \shaThree\ algorithm and the most complex part.
The proof relies of the use of overrides.
\saw\ proves the \cryptol\ specifications of the five inner \keccak\ functions equivalent to their C counterparts and replaces them in the C code with the \sawcore\ from the \cryptol.
With the overrides, \saw\ proves the equivalence for \keccak\ in \cryptol\ and C over any input.

Future work is to prove out the equivalence of the input padding and the squeeze part of the \shaThree\ computation.
These proofs are complicated by the typing in \cryptol\ and the layout of bytes in C.
There is also some effort to show the equivalence for the other digest sizes.
Other work looks to accomplish a proof of a Dafny implementation of \shaThree.
Dafny synthesizes to several different backend languages making a proven \shaThree\ implementation very interesting.
