Initially it appeared that visual inspection with Cryptol's standard syntax was subpar.
The language's functional folds and list comprehensions do not align easily with the imperative and array-based NIST standard. 
By leveraging Cryptol's versatility, utility functions can visually mimic the sequential nature of the NIST specification. 

\subsection{For Utility Function}
The phrase \emph{for all} indicates a looping structure and is quite common in the NIST standard. 
It is used in the definition of the pi function shown in Figure \ref{fig:nistPi}.

\begin{figure}[h]
  \centering
  \begin{lstlisting}[basewidth = {.5em},basicstyle={\scriptsize}]
1. For all triples (x,y,z) such that 
  0 <= x < 5, 0 <= y < 5, and 0 <= z < w,
  let A'[x, y, z] = A[(x + 3y) mod 5, x, z].
2. Return A'.
  \end{lstlisting}
  \caption{NIST Pi definition}
  \label{fig:nistPi}
\end{figure}
    
By nesting Cryptol's list comprehensions, the pi function can be easily implemented.
This is illustrated in Figure \ref{fig:cryptolPi}. 

\begin{figure}[h]
  \centering
\begin{lstlisting}[language=Cryptol]
type COL = 5 
type W = 64
type STATE_ARR = [COL][COL][W]
LIST4 = [0..COL - 1]
LIST63 = [0..W - 1]
pi : STATE_ARR -> STATE_ARR
pi a = [[[a @x @((x + 3*y) % 'COL) @z 
            | z <- LIST63] 
          | y <- LIST4] 
        | x <- LIST4]
\end{lstlisting}
\caption{Cryptol Pi definition}
\label{fig:cryptolPi}
\end{figure}

Cryptol declares list comprehension variables after they are used, and the order of the nested loops is not as apparent. 
Using a helper function, \emph{for}, a clearer definition seen in Figure \ref{fig:cryptolAmendedPi}.

\begin{figure}[h]
  \centering
\begin{lstlisting}[language=Cryptol]
type COL = 5 
type W = 64
type STATE_ARR = [COL][COL][W]
LIST4 = [0..COL - 1]
LIST63 = [0..W - 1]
pi : STATE_ARR -> STATE_ARR
pi a = for LIST4 (&x ->
          for LIST4 (&y -> 
            for LIST4 (&z -> 
              a @x @((x + 3*y) % 'COL) @z)))
\end{lstlisting}
\caption{Amended Cryptol Pi definition}
\label{fig:cryptolAmendedPi}
\end{figure}

The \emph{for} utility function follows the same intuition as the implementation in most imperative languages in that it iterates and runs a loop body over defined elements. 
It recieves a function and a list of arguments and returns a list of results from calling the function on every element of the input array. 
Figure \ref{fig:cryptolFor} shows it's implementation.

\begin{figure}[h]
  \centering
\begin{lstlisting}[language=Cryptol]
for : {n, a, b} [n]a -> (a -> b) -> [n]b
for vals loop = [loop i | i <- vals]
\end{lstlisting}
\caption{Cryptol For utility function}
\label{fig:cryptolFor}
\end{figure}

Though it is simple to implement, it reorders and structures the arguments in a very elegant way.
In constrast with standard cryptol syntax, the list to iterate over and the iterating variable are visible before the loop body. 
List comprehensions are hidden away behind this function call and visual equivalence is clearer.

\subsection{While Utility Function}
Another common phrasing is the \emph{let...for...let}. 
This defines a state upon which a folding occurs, iteratively condensing the information into a final state. 
Imperatively, it is most comparable to a while loop. 
As shown by Figure \ref{fig:nistRC}, is present in the rc function defined by the NIST specification.

\begin{figure}[h]
  \centering
\begin{lstlisting}[basewidth = {.5em},basicstyle={\scriptsize}]
1. If t mod 255 = 0, return 1.
2. Let R = 10000000.
3. For i from 1 to t mod 255, let:
  a. R = 0 || R;
  b. R[0] = R[0] ^ R[8];
  c. R[4] = R[4] ^ R[8];
  d. R[5] = R[5] ^ R[8];
  e. R[6] = R[6] ^ R[8];
  f. R = Trunc8[R].
4. Return R[0].
\end{lstlisting}
\caption{NIST RC definition}
\label{fig:nistRC}
\end{figure}

Maintaining this state through Cryptol's list comprehensions can be difficult.
The comprehensions require a list to iterate through. 
Since the \emph{for} utility function is simply a wrapper for list comprehensions, it is equally unsuited for visual verification on such steps. 
The number of iterations in the RC function is bounded by [1, 254] by its modular condition.
This range provides the list to iterate over as required by Cryptol list comprehensions. 
From these observations, a rather ungainly Cryptol definition results.
This definition is shown in Figure \ref{fig:cryptolRC}.

\begin{figure}[h]
  \centering
\begin{lstlisting}[language=Cryptol]
rc : [64] -> Bit
rc t = if (t % 255) == 0 
  then True 
  else rs !0 @0 where
    rs = [128:[8]] # [
      if i <= t % 255
        then (take'{8} ([
          (r@0) ^ (r@8),
          r@1,
          r@2,
          r@3, 
          (r@4) ^ (r@8),
          (r@5) ^ (r@8),
          (r@6) ^ (r@8),
          r@7,
          r@8])
            where r = [False] # (rs @(i - 1)))
        else rs @(i - 1)
      | i <- [1..254]]
\end{lstlisting}
\caption{Cryptol RC definition}
\label{fig:cryptolRC}
\end{figure}

Not only is this definition computationally slower as it runs 255 loops every time it is called, it also awkwardly indexes into a list of previous states. 
This differs vastly from the NIST specificiation both visually and functionally. 
The \emph{while} utility function, accepts an initial state, condition function, and body function to recursively loop. 
It consolodates the information of the previous state until the condition is sastified. 
This definition is found in Figure \ref{fig:cryptolamendedRC}.

\begin{figure}[h]
  \centering
\begin{lstlisting}[language=Cryptol]
rc : [64] -> Bit
rc t = if (t % 255) == 0 
  then True 
  else (while {i = 1, R = 128:[8]}
    (&state -> state.i <= t % 255)
    (&state -> {
      i = state.i + 1, 
      R = (take'{8} ([
        (r@0) ^ (r@8),
        r@1,
        r@2,
        r@3,
        (r@4) ^ (r@8),
        (r@5) ^ (r@8),
        (r@6) ^ (r@8),
        r@7, 
        r@8]) 
          where r = [False] # state.R)})
    ).R @0
\end{lstlisting}
\caption{Amended Cryptol RC definition}
\label{fig:cryptolamendedRC}
\end{figure}

The final state is returned by the \emph{while} function. 
This iterates only until the condition is met, avoids the clutter of indexing within a larger array of all previous states, and places the condition in a more obvious location before the loop body. 
This utility function is defined in Figure \ref{fig:cryptolWhile}.

\begin{figure}[h]
  \centering
\begin{lstlisting}[language=Cryptol]
while : {a} a -> (a -> Bit) -> (a -> a) -> a
while state cond f = if (cond state)
  then (while (f state) cond f)	
  else state
\end{lstlisting}
\caption{Cryptol While utility function}
\label{fig:cryptolWhile}
\end{figure}