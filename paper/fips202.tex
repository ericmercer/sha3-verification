This section illustrates with a few examples how the \fips\ specification is captured in \cryptol. 
\cryptol's list comprehensions do not directly align with the quantification and iteration in the specification. 
This misalignment makes the \cryptol\ specification more difficult to read and to manually argue its equivalence relative to the published specification.
The two specifications are better aligned through a library of two methods in \cryptol\ that hide the list comprehensions so that the \cryptol\ specification reads more like the \fips\ specification. The two library methods are the \emph{for-method} and the \emph{while-method} discussed in this section.

\subsection{The for-method}

\newsavebox{\fipsPi}
\begin{lrbox}{\fipsPi}
  \begin{lstlisting}[basewidth = {.5em},basicstyle={\small}]
    1. For all triples (x,y,z) such that 
           0 <= x < 5, 0 <= y < 5, and 0 <= z < w,
       let A'[x, y, z] = A[(x + 3y) mod 5, x, z].
    2. Return A'.
  \end{lstlisting}
\end{lrbox}

\newsavebox{\PiCry}
\begin{lrbox}{\PiCry}
  \begin{lstlisting}[language=Cryptol, numbers=left, escapeinside={;}{;}]
    type STATE_ARR = [5][5][64];\label{line:pistate};
    pi : STATE_ARR -> STATE_ARR
    pi a = [  [  [a @x @((x + 3*y) % 5) @z;\label{line:pi};
                 | z <- [0..63]];\label{line:list};
              | y <- [0..4]] 
           | x <- [0..4]]
  \end{lstlisting}
\end{lrbox}

\newsavebox{\PiCryLib}
\begin{lrbox}{\PiCryLib}
  \begin{lstlisting}[language=Cryptol]
    pi a = for [0..4] (&x ->
             for [0..4] (&y -> 
               for [0..63] (&z -> 
                 a @x @((x + 3*y) % 5) @z)))
  \end{lstlisting}
\end{lrbox}

\newsavebox{\formethod}
\begin{lrbox}{\formethod}
  \begin{lstlisting}[language=Cryptol]
    for : {n, a, b} [n]a -> (a -> b) -> [n]b
    for vals f = [f i | i <- vals]
  \end{lstlisting}
\end{lrbox}

\begin{figure}[t]
  \begin{center}
    \begin{tabular}{l}
      \usebox{\fipsPi} \\ \\
      \multicolumn{1}{c}{(a)} \\ \\
      \usebox{\PiCry} \\ \\
      \multicolumn{1}{c}{(b)} \\ \\
      \usebox{\formethod} \\ \\
      \multicolumn{1}{c}{(c)} \\ \\ 
      \usebox{\PiCryLib} \\ \\
      \multicolumn{1}{c}{(d)}
    \end{tabular}
  \end{center}
  \caption{The $\pi$ function. (a) The \fips\ specification. (b) The \cryptol\ with comprehensions. (c) The for-method definition. (d) The \cryptol\ with the for-method.}
  \label{fig:pi}
\end{figure}

A common idiom in the \fips\ publication is using quantification over a finite domain defined by linear constraints to define input for some computation.
The definition of the $\pi$ function for \keccak\ shown in \figref{fig:pi}(a) is one such example.
Here the quantification is used to define the set of indices that are part of the $\pi$ computation that transforms the state array. The intent is that of a \emph{parallel-for} where order does not matter since the new state array $A^\prime$ only depends on the old state array $A$.

The \cryptol\ definition using list comprehensions is given in \figref{fig:pi}(b).
\lineref{line:pistate} defines the state as a $5 \times 5$ array of 64-bit words.
\lineref{line:pi} starts the definition of $\pi$. 
Here the comprehensions are nested with the list indices appearing in reverse order to follow the nesting.
\lineref{line:list} gives the domain for $z$ as a list with values from 0 to 63.
The domains for $x$ and $y$ are given similarly.
The domains for $x$, $y$, and $z$ all match the domains defined in the quantification in \figref{fig:pi}(a).
Although the list comprehensions are not unreadable, the \cryptol\ can be improved by creating a for-method to hide some of the details.

\figref{fig:pi}(c) is the \cryptol\ definition of the for-method.
It takes two arguments: a list, \emph{vals}, for the comprehension, and a function, $f$, to apply to each element of the list.
Its definition is as expected: it creates a new list that is the result of applying $f$ to each element in \emph{vals}.

\figref{fig:pi}(d) is the rewritten \cryptol\ using the for-method.
It still requires the indices lists for the list comprehensions, but it has a more obvious correspondence to the original definition in \figref{fig:pi}(a).
The for-method reorders and structures the arguments in a very elegant way.
It hides the list comprehensions to simplify the task of visual inspection for equivalence.

\subsection{The while-method}

\newsavebox{\fipsRc}
\begin{lrbox}{\fipsRc}
  \begin{lstlisting}[basewidth = {.5em},basicstyle={\small}]
    1. If t mod 255 = 0, return 1.
    2. Let R = 10000000.
    3. For i from 1 to t mod 255, let:
      a. R = 0 || R;
      b. R[0] = R[0] ^ R[8];
      c. R[4] = R[4] ^ R[8];
      d. R[5] = R[5] ^ R[8];
      e. R[6] = R[6] ^ R[8];
      f. R = Trunc8[R].
    4. Return R[0].
  \end{lstlisting}
\end{lrbox}

\newsavebox{\RcCry}
\begin{lrbox}{\RcCry}
  \begin{lstlisting}[language=Cryptol, numbers=left, escapeinside={;}{;}]
    rc : [64] -> Bit
    rc t =  if (t % 255) == 0 then 1
            else rs !0 @0 where;\label{line:return};
              rs = [0b10000000] #;\label{line:init};
                   [if i <= t % 255 then;\label{line:compute};
                      (take'{8} ([(r@0) ^ (r@8),
                                   r@1, r@2, r@3, 
                                  (r@4) ^ (r@8),
                                  (r@5) ^ (r@8),
                                  (r@6) ^ (r@8),
                                   r@7,r@8])
                        where r = [0] # (rs @(i - 1)))
                    else rs @(i - 1);\label{line:stutter};
                   | i <- [1..254]];\label{line:alliters};
  \end{lstlisting}
\end{lrbox}

\newsavebox{\whilemethod}
\begin{lrbox}{\whilemethod}
  \begin{lstlisting}[language=Cryptol]
    while : {a} a -> (a -> Bit) -> (a -> a) -> a
    while state cond f = 
      if (cond state) then (while (f state) cond f)	
      else state
  \end{lstlisting}
\end{lrbox}

\newsavebox{\RcCryLib}
\begin{lrbox}{\RcCryLib}
  \begin{lstlisting}[language=Cryptol, numbers=left, escapeinside={;}{;}]
    rc : [64] -> Bit
    rc t =  if (t % 255) == 0 then 1 
            else (while {i = 1, R = 0b10000000};\label{line:state};
                        (&state -> state.i <= t % 255);\label{line:cond};
                        (&state -> {
                          i = state.i + 1,;\label{line:inc};
                          R = (take'{8} ([(r@0) ^ (r@8),
                                           r@1, r@2, r@3,
                                          (r@4) ^ (r@8),
                                          (r@5) ^ (r@8),
                                          (r@6) ^ (r@8),
                                           r@7, r@8]) 
                                where r = [0] # state.R)})
                 ).R @0;\label{line:libreturn};
  \end{lstlisting}
\end{lrbox}

\begin{figure}[t]
  \begin{center}
    \begin{tabular}{l}
      \usebox{\fipsRc}
    \end{tabular}
  \end{center}
  \caption{The \fips\ specification for \emph{rc}.}
  \label{fig:rc}
\end{figure}

Another common idiom in the \fips\ standard is to iteratively transform some state a fixed number of times to arrive at a final state.
The definition of the \emph{round constant}, \emph{rc}, function in \figref{fig:rc} is one such example.
The \emph{rc} function is a support function for the $\iota$ function in \keccak\ and is used to build a mask.

The \emph{rc} function checks the value of the input integer $t$, and returns the bit $0$ if it is a multiple of $255$.
If not, then Line 2 defines $R$ as the 8-bit value $128$, and Line 3 iteratively transforms $R$ a fixed number of times: $t\ \mathbf{mod}\ 255$.
The transform is defined on Line 3a through Line 3f.
Line 3a prepends a $0$ bit to $R$ so that it is now a 9-bit value. The array indexing refers to bit positions from left to right so $R[0]$ is the prepended left-most bit and $R[8]$ is the right-most bit. Each transform on $R$ prepends the 0 (Line 3a), does an exclusive-or with specific bit positions (Line 3b through Line 3e), and then takes the left-most 8 bits for the next round effectively dropping the right-most bit (Line 3f).
It returns the $R[0]$ bit at the end.

The \cryptol\ equivalent for \emph{rc} in \figref{fig:rccry}(a) is considerably more complex with the list comprehensions.
Returning $1$ when the input is a multiple of $255$ is straightforward.
It is when that is not the case that it is less obvious.
The list comprehension builds a list of $255$ 8-bit values.
\lineref{line:return} returns the bit at index 0 ($@0$) from the last 8-bit value in the array ($!0$).
The computation of the list of 8-bit values follows.

\lineref{line:init} is the first entry in the list and it is the initial value of $R$.
The following entries depend on the value of the input $t$ and the value of $i$ that comes from the list used in the list comprehension on \lineref{line:alliters}.
\lineref{line:compute} checks if $i$ is within the number of iterations, and if it is, then the new entry in the list is computed from the previous entry as defined by the transform in \figref{fig:rc}.
\lineref{line:stutter} is the case for when $i$ is not within the number of iterations.
Here the previous value in the list is copied to stutter the last value of $R$. 
The \cryptol\ specification creates all $255$ entries in the list regardless of the input value $t$ except for the case when $t$ is a multiple of 255.
As before, \figref{fig:rccry}(a) is not altogether unreadable, but it can be improved considerably.

\figref{fig:rccry}(b) is a recursive definition of a \emph{while-loop}.
It takes as input a state, a predicate on the state, \emph{cond}, that is true when it should loop and false otherwise, and a function, $f$, that computes a new state from the old state on each iteration.
The definition makes recursive calls to the while-method until the state predicate is false at which point it returns the current value of \emph{state}.

\begin{figure}[h]
  \begin{center}
    \begin{tabular}{l}
      \usebox{\RcCry} \\ \\
      \multicolumn{1}{c}{(a)} \\ \\
      \usebox{\whilemethod} \\ \\
      \multicolumn{1}{c}{(b)} \\ \\ 
      \usebox{\RcCryLib} \\ \\
      \multicolumn{1}{c}{(c)}
    \end{tabular}
  \end{center}
  \caption{The \emph{rc} function in \cryptol. (b) The \cryptol\ with list comprehensions. (b) The recursive while-method definition. (c) The \cryptol\ with the while-method.}
  \label{fig:rccry}
\end{figure}

\figref{fig:rccry}(c) is the definition of \emph{rc} using the while-method in \figref{fig:rccry}(b).
\lineref{line:state} defines the state for the while loop to be both an integer $i$ for a counter and the 8-bit value $R$.
\lineref{line:cond} is the looping condition defined on the state.
The body of the loop not only computes the new value of $R$ but \lineref{line:inc} increments $i$ to track the loop iteration.
The return value is given on \lineref{line:libreturn} as the $0^\mathrm{th}$ bit from the last value of $R$.
Here, $R$ is computed only the number of times required by $t$.
This change differs significantly from the definition without the while-method.
As with the for-method, the while-method reduces the gap between how things are defined in \fips\ and how they are defined in \cryptol.
