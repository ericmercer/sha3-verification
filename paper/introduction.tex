This paper discusses an automated proof of the \emph{Secure Hash Algorithm 3} (\shaThree) in \openssl\ using the \emph{Software Analysis Workbench} by Galois.
\shaThree\ is the latest algorithm for computing \emph{digests}, or hashes, from data to be published by the \emph{National Institute of Standards and Technology} (\nist).
The algorithm is the successor to \shaTwo.
Its intent is to address weaknesses in \shaTwo, and it departs from its predecessors in that it is based on the \keccak\ family of cryptographic primitives \cite{fips202}.

\openssl\ is a commercial grade free cryptographic library \cite{openssl}.
Its implementation of the \emph{Secure Sockets Layer} (SSL) and \emph{Transport Layer Security} protocols are widely deployed on internet facing applications.
Computing digests is a critical operation in these protocols and is the basis to the security guarantees provide by these protocols.
Widely deployed implementations, such as \openssl, are especially important to be correct because of the available attack surface to hackers if a vulnerability where discovered in such implementations.
The seriousness of getting these implementations correct is recently seen in the Apache \texttt{log4j} vulnerability affecting virtually all internet facing servers \cite{log4j}.

The \emph{Software Analysis Workbench} (\saw) is a formal verification tool to prove equivalence between \sawcore\ models. \sawcore\ models are defined in the \cryptol\ functional language or extracted from C or Java with \emph{symbolic execution}.
\cryptol\ itself is designed to specify cryptographic primitives at the bit level.
\saw\ has been shown capable in proving implementation of cryptographic primitives equivalent to hardware and software implementations \cite{crypt-hi,hard-soft,design-verif}.
Recent work emphasizes not just the utility of \saw, but the importance of modular decomposition of cryptographic primitives in their definition as that decomposition lends itself to efficient verification in \saw \cite{nfm-us}. 

This paper details another example of \saw\ verification only this time it is a proof showing the \openssl\ C implementation of \shaThree\ matches a \cryptol\ specification derived from the published \fips\ standard by \nist\ \cite{fips202}.
The \cryptol\ specifications, \saw\ scripts, and extracted \sawcore\ model from the \openssl\ C implementation are documented, and available, in a public GitHub repository for independent verification \cite{repo}.
The proof relies on two specifications, both created as part of this research, of the \shaThree\ algorithm: one that closely matches the published standard and another incorporating features in the C implementation.
\saw\ shows an input to output equivalence between these two specifications for the 256 bit digest over a fixed set of message sizes. The proof takes around 60 hours of computation time.

The proof then turns to the inner \keccak\ function that computes the actual bits for the digest.
Here it shows that the C implementation exactly matches the \cryptol\ specification on any input.
The proof requires the use of \emph{overrides} in \saw\ to simplify the extracted model from C after symbolic execution.
The proof of equivalence between the \keccak\ in C and \cryptol\ takes around 15 minutes of computation time.

The proof, with its artifacts in the published repository, make the following contributions:
\begin{compactitem}
  \item \cryptol\ for \shaThree\ that passes the test vectors and correlates with \fips.
  \item \cryptol\ for \shaThree\ that uses a reordered memory layout and precomputed tables that correlate with C implementations.
  \item A \saw\ proof showing the two specifications equivalent for the 256 bit digest over a series of message sizes.
  \item A \saw\ proof that the \keccak\ specification in \cryptol\ is equivalent to the \openssl\ C implementation of the same function for all inputs.
\end{compactitem}
The proof shows that \saw\ is capable of proving equivalence to C implementations of modern cryptographic primitives.
It also shows that the modular definition of \shaThree\ naturally lends itself to formal verification in \saw.
The work in its entirety argues the importance of using formal languages such as \cryptol\ for the specification of cryptographic primitives rather than English prose and math.
It also argues the necessity, and viability, of proving at the intermediate representation level equivalence between cryptographic specifications and actual C implementations.

\begin{comment}
\noindent \textbf{Limitations}
\begin{compactitem}
  \item Visual inspection and limited test vectors for (1) and (2)
  \item Only provide proof certificates for message sizes of X to Y for digest sizes of \textbf{FIXME} for (3)
  \item Visual inspection only for anything above Keccak for (3) in OpenSSL's C-code
\end{compactitem}

\subsection{Related Work}
  Previously, Galois' Inc. worked with Amazon to prove out parts of Amazon's s2n HMAC code to the NIST specification.  
In their work they stubbed out use of the SHA2 algorithm because their HMAC allows for customization of the algorithm type and implementation.

The work by Decker et. al, extended on Galois' work by choosing a single SHA2 implementation to prove out.  
They chose OpenSSL's SHA 256 implementation, as that is a commonly used library (and one that is usable in s2n).  
They were able to prove out the inner hash computations of that implementation using a methodology of compositional proofs to do so.  
They were also able to connect up their proof with the code used by Galois' to stub out the SHA2 algorithm, which implies that the functional equivalence property established with the OpenSSL code applies to Galois' Cryptol implementation as well.  
Their work showed the viability and limitations of creating proofs of functional equivalence for legacy cryptographic implementations.



\subsection{SHA 3}
This work shows that this same methodology can be extended to modern cryptographic algorithms.
The SHA 3 family of algorithms is defined by NIST in the FIPS 202 publication.
It specifies a block hashing algorithm, Keccak, that is used by each of the different SHA and SHAKE algorithms.
In comparison to the SHA 2 family of algorithms, Keccak features significantly more complex and robust computations.
This increase in computation was of interest, because it presses the limits of SAW's capabilities.

\subsection{Results}
This work shows a complete proof of functional equivalence from a verified implementation of the NIST standard to OpenSSL's C code for the Keccak function.
In doing so, it shows SAW's ability to stand up against more difficult computations.
It also provides insights into using Cryptol's functional nature to further simplify the step of visually verifying Cryptol code to the NIST standard.

Because of how NIST defines the standard, the keccak function is used for each digest size of the SHA3 algorithm, with the only significant differences being in the padding/sponging parts of the algorithm, which falls outside this work's scope.
This means that the proof of correctness for this function extends usefulness beyond just the SHA3-256 level which we focused on.
In our repository, we wrote the Cryptol specification to be able to run any of the SHA3 family of algorithms using the same keccak.
A simple extension of this work would be to take that already created and verified Cryptol specification, and prove it up against OpenSSL's slightly modified keccak versions.
\end{comment}