\section{SHA3 OpenSSL in Cryptol}\label{sec:openssl}
NIST specification invokes the rc method seven times for each iota function call.
As the rc inputs are the same for each round, OpenSSL's optimized implementation of keccak stores 
the output values as a constant table. Initially, we matched out Cryptol execution of the NIST standard 
as closely as possible--including rc method calls in the iota function. We soon found that SAW was unable 
to map our implementation of keccak to OpenSSL's without the use of a constant table.
Our final Cryptol implementation uses the rc definition found in the NIST specification to generate a table 
that is easiy compared to OpenSSL. This allows SAW to generate a proof and permits visual inspection of equivalence.