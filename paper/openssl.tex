\section{SHA3 OpenSSL in Cryptol}\label{sec:openssl}
The NIST specification invokes the rc method seven times for each iota function call.
As the rc inputs are the same for each round, OpenSSL's optimized implementation of keccak stores the output values as a constant table rather than repeatedly compute them. 
Initially, the Cryptol code to the NIST standard as closely as possible\textemdash including RC method calls in the iota function. 
SAW was unable to map such an implementation of keccak to OpenSSL's implementation without the use of a constant table. 
The final Cryptol implementation uses the RC definition found in the NIST specification to generate a table that is easiy compared to OpenSSL. 
This allows SAW to generate a proof and permits visual inspection of equivalence.