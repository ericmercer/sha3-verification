This section details the key differences between the \openssl\ implementation of \shaThree\ and the \fips\ specification that prevented a direct proof of equivalence between the \cryptol\ and C code.
These differences are expressed in a second \cryptol\ model derived from the \fips\ model discussed in the previous section.
\saw\ proves these two models equivalent for the 256-bit digest size as discussed in \secref{sec:proof}.
This equivalence considers the whole of the sponge construction algorithm proving that the digest is the same from each of them.
That said, and as a reminder, the equivalence between the \keccak\ description in \cryptol\ and the C implementation in \openssl\ holds for any input message and any digest size.
That proof uses the \cryptol\ definition for the model discussed in this section that includes all the differences seen in \openssl, and the overrides used in that proof also come from the \cryptol\ discussed in this section.

\subsection{State Array Structure and Computation}

The first set of differences in the \openssl\ \shaThree\ implementation is in the structure of the state array and how it operates on that state array.
The difference in structure is an artifact of how C maps arrays to memory.
The \fips\ structures the state array as a $5 \times 5$ grid of 64-bit words with each 64-bit word being a \emph{lane}.
It assumes the layout of the data follows a normal cartesian three dimensional coordinate system with $x$ being the horizontal axis, $y$ being the vertical axis, and $z$ being the depth on a lane to access an individual bit.
For a state $A$, $A[x,y,z]$ accesses the $z^\mathrm{th}$ bit from the 64-bit word on the $x^\mathrm{th}$ column and the $y^\mathrm{th}$ row.

\openssl\ declares the state as follows: \texttt{uint64\textunderscore t A[5][5]}.
The C standard stores multidimensional arrays in contiguous memory in \emph{row-major} order meaning that each row of five 64-bit values appear consecutively in memory.
Indexing the multidimensional array follows the standard mathematical definition for indexing matrices: $A[x,y]$ is the element at the $x^\mathrm{th}$ row and $y^\mathrm{th}$ column.
This meaning is just opposite of that in the standard.

Adding to the complexity is that the C standard does not provide array indexing to get a bit from a value.
For the $A[x,y,z]$ example, there is no array bracket notation to get the $z^\mathrm{th}$ bit in a 64-bit word, so notation in the \fips\ standard such as that seen in the definition of the $rc$ function in \figref{fig:rc}(a) has no direct analogue in C.
The consequence is that the \openssl\ implementation does everything at the level of the lanes, operating on each lane as a 64-bit entity, and it never refers to an individual bit in a lane.
Finally, the bit ordering in the lanes in the \fips\ standard is just opposite the ordering in C meaning that the direction of shifting in the standard is opposite the direction used in the C implementation.

The \cryptol\ for the \fips\ model is rewritten to reflect the memory layout differences but leaves in place the bit-level operations.
The $x$ and $y$ indexing is swapped, and the shifts are reversed.
For reference, \figref{fig:piopenssl} is the rewritten $\pi$ function in \cryptol\ than matches swapping of the indices in \openssl (compare to \figref{fig:pi}(d)).

\newsavebox{\PiOpenSSL}
\begin{lrbox}{\PiOpenSSL}
  \begin{lstlisting}[language=Cryptol]
    pi a = for [0..4] (&y ->
             for [0..4] (&x -> 
               for [0..63] (&z -> 
                 a @x @((x + 3*y) % 5) @z)))  
  \end{lstlisting}
\end{lrbox}

\begin{figure}[t]
  \begin{center}
    \usebox{\PiOpenSSL}
  \end{center}
  \caption{The \openssl\ implementation of $\pi$ that reverses indexes.}
  \label{fig:piopenssl}
\end{figure}

\subsection{Constant Lookup Tables}

\newsavebox{\ciota}
\begin{lrbox}{\ciota}
  \begin{lstlisting}
    void Iota(uint64_t A[5][5], size_t i) {
      //assert(i < (sizeof(iotas) / sizeof(iotas[0])));
      if (i < (sizeof(iotas) / sizeof(iotas[0]))) {
        A[0][0] ^= iotas[i];
      }
    }
  \end{lstlisting}    
\end{lrbox}

% iota : STATE_ARR -> [64] -> STATE_ARR
% iota a i = for [0..4] (\y ->
%               for [0..4] (\x ->
%                   for [0..63] (\z ->
%                       if ((x == 0) && (y == 0))
%                           then (a @0 @0 @z) ^ (LISTIOTAS @i @z)
%                       else a @y @x @z)))

\newsavebox{\cryciota}
\begin{lrbox}{\cryciota}
  \begin{lstlisting}[language=Cryptol]
    iota a i =  for LIST4 (\y ->
                  for LIST4 (\x ->
                    for LIST63 (\z ->
                      if ((x == 0) && (y == 0))
                        then (a @0 @0 @z) ^ (LISTIOTAS @i @z)
	                      else a @y @x @z)))

    LISTIOTAS = [reverse (
      (while {RC = 0:[W], j = 0} 
              (\state -> state.j <= `L) 
              (\state -> { 
                RC = for [0..63] 
                  (\z -> if z == index then rc (state.j + 7*i)
                          else state.RC @z)
                    where index = ((1:[8]) << state.j) - 1,
                                    j = state.j + 1})
      ).RC) | i <- LISTROUNDS:[_][64]]
  \end{lstlisting}    
\end{lrbox}

\begin{figure}[t]
  \begin{center}
    \begin{tabular}{l}
      \usebox{\cryciota} \\ \\
    \end{tabular}
  \end{center}
  \caption{The computed \emph{rc} table with the lookup in $\iota$.}
  \label{fig:iota}
\end{figure}

The second difference is in hardcoding lookup Tables for constants rather than computing the constants along the way.
The \fips\ \cryptol\ specification computes each constant on-the-fly as needed in each round.
This redundant computation is seen with the rounding constants, \emph{rc}, for $\iota$, and the rotate constants for $\rho$. 
The \openssl\ implementation hardcodes these constants in tables without the provenance of the computation and looks up the constants in the tables.
The new \cryptol\ specification for the \openssl\ uses the same lookup tables, but it differs from the \openssl\ implementation in that it computes the lookup tables once so that the values in the table now have provenance.
This provenance is sufficient for the equivalence proof with the C code.
\figref{fig:iota} shows the $\iota$ function that uses the computed lookup table.

The change to the state array structure and the added provenance giving the computation for lookup constants are the only changes to the \cryptol\ from the \fips.
These changes are sufficient for \saw\ to prove the input to output equivalence between the two \cryptol\ specifications for a fixed set of message sizes.
And they are sufficient to prove the equivalence to the actual C code in \openssl\ in the \keccak function for any arbitrary state.
The result adds more evidence to the ability to prove out implementations of cryptographic primitives when given specifications. 