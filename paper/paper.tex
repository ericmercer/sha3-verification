\documentclass[runningheads]{llncs}
%
\usepackage{paralist}
\usepackage{comment}
\usepackage{hyperref}
\usepackage{graphicx}
\usepackage{algorithm}
\usepackage{algpseudocode}
\usepackage{color}
\definecolor{blue}{RGB}{42,0,255}
\definecolor{green}{RGB}{63,127,95}
\definecolor{purple}{RGB}{127,0,85}
\definecolor{lime}{RGB}{114, 148, 0}
\usepackage{listings}
\lstdefinelanguage{Cryptol}{
  keywordstyle=[1]\color{blue}\bfseries,
  keywordstyle=[2]\color{purple}\bfseries,
  keywordstyle=[3]\color{black}\bfseries,
  basicstyle={\color{lime}\scriptsize},
  literate=*
  {*}{{{\color{black}$*$}}}{1}
  {=}{{{\color{black}=}}}{1}
  {+}{{{\color{black}+}}}{1}
  {-}{{{\color{black}$-$}}}{1}
  {\%}{{{\color{black}\%}}}{1}
  {(}{{{\color{black}(}}}{1}
  {)}{{{\color{black})}}}{1}
  {\{}{{{\color{black}\{}}}{1}
  {\}}{{{\color{black}\}}}}{1}
  {[}{{{\color{black}[}}}{1}
  {]}{{{\color{black}]}}}{1}
  {|}{{{\color{black}$|$}}}{1}
  {>}{{{\color{black}$>$}}}{1}
  {<}{{{\color{black}$<$}}}{1}
  {\#}{{{\color{black}\#}}}{1}
  {@}{{{\color{black}@}}}{1}
  {.}{{{\color{black}.}}}{1}
  {\'}{{{\color{black}'}}}{1}
  {:}{{{\color{black}:}}}{1}
  {^}{{{\color{black}$\char`\^$}}}{1}
  {&}{{{\color{black}\textbackslash}}}{1},
  morekeywords=[1]{Bit, Integer, STATE_ARR, W, COL, type, LIST4, LIST63, True, False},
  morekeywords=[2]{if, then, else, where},
  morekeywords=[3]{pi, rc, cond, a, b, x, y, z, for, n, r, R, f, rs, i, index, t, 
    vals, loop, state, while, take}
}

\definecolor{dkgreen}{rgb}{0,0.6,0}
\definecolor{almond}{rgb}{0.94, 0.87, 0.8}
\definecolor{amaranth}{rgb}{0.9, 0.17, 0.31}
\definecolor{azure}{rgb}{0.0, 0.5, 1.0}
\definecolor{gray}{rgb}{0.5,0.5,0.5}

\lstdefinestyle{customc}{
  frame=none,
  language=C,
  basicstyle=\footnotesize\ttfamily, 
  aboveskip=3mm,
  belowskip=3mm,
  showstringspaces=false,
  numbers=left,
  keywordstyle=\bfseries\color{azure},
  commentstyle=\itshape\color{dkgreen},
  identifierstyle=\color{black},
  stringstyle=\color{almond},
  numbersep=-1pt,
  numberstyle=\color{gray},
  breaklines=true,
  breakatwhitespace=true,
  tabsize=3,
}

% Used for displaying a sample figure. If possible, figure files should
% be included in EPS format.
%
% If you use the hyperref package, please uncomment the following line
% to display URLs in blue roman font according to Springer's eBook style:
\renewcommand{\UrlFont}{\color{blue}\rmfamily}
\newcommand{\keccak}{\textsc{Keccak}}
\newcommand{\shaThree}{\texttt{SHA-3}}
\newcommand{\shaTwo}{\texttt{SHA-2}}
\newcommand{\shaOne}{\texttt{SHA-1}}
\newcommand{\nist}{\texttt{NIST}}
\newcommand{\fips}{\texttt{FIPS 202}}
\newcommand{\saw}{\texttt{SAW}}
\newcommand{\sawcore}{SAWCore}
\newcommand{\cryptol}{Cryptol}
\newcommand{\openssl}{OpenSSL}
\newcommand{\figref}[1]{Fig.~\ref{#1}}
\newcommand{\algoref}[1]{Alg.~\ref{#1}}
\newcommand{\lineref}[1]{Line~\ref{#1}}
\newcommand{\secref}[1]{Sec.~\ref{#1}}
\newcommand{\tableref}[1]{Table~\ref{#1}}


\begin{document}
%
\title{Verifying the \shaThree\ Implementation from \openssl\ with the Software Analysis Workbench}
%
%\titlerunning{Abbreviated paper title}
% If the paper title is too long for the running head, you can set
% an abbreviated paper title here
%
\author{
  Parker Hanson\inst{1} \and
  Benjamin Winters\inst{1} \and
  Eric Mercer\inst{1}\orcidID{0000-0002-2264-2958} \and
  Brett Decker\inst{1}
}
%
\authorrunning{P. Hanson et al.}
% First names are abbreviated in the running head.
% If there are more than two authors, 'et al.' is used.
%
\institute{
  Brigham Young University, Provo UT 84602, USA
}
%
\maketitle              % typeset the header of the contribution
%
\begin{abstract}
\shaThree\ is the new standard from \nist\ for computing digests. 
\openssl\ is one of the most widely used implementations of cryptographic protocols.
\cryptol\ and the Software Analysis Workbench (\saw) are a language and verification engine to specify cryptographic primitives and prove equivalence to implementations.
As digests are the basis for all cryptographic protocols, the research in this paper is an automated \saw\ proof of equivalence between the \shaThree\ standard and the actual \openssl\ implementation in C.
The research contributes \cryptol\ specifications for the published standard and the \openssl\ implementation of the standard with its several changes.
The equivalence proof shows the importance of overrides in \saw\ that replace C code with \cryptol\ to reduce the complexity of the proof obligations that must be accomplished relative to the implementation in C.
The research establishes the viability of verifying modern cryptographic primitives and the importance of modularity in their definitions, and implementations, for overrides.

% The \keccak\ function in the SHA-3 NIST standard is responsible for the major computations of all SHA-3 hash constructs.
% These constructs form the basis for modern secure hashing.
% This work extends and improves the methodology used to verify Amazon's s2n HMAC and OpenSSL's SHA2-256 inner hash function. 
% It uses the Software Analysis Workbench (SAW) to prove out functional equivalence between OpenSSL's SHA-3 Keccak function and a verified implementation of SHA-3 in Cryptol, a domain-specific language for cryptography. 
% SAW allows for a modular approach to writing proofs through the use of contracts, which define a functions input to output relationship.
% Each contract, also called an override, is an obligation in the full proof.
% This work leverages SAW by creating overrides to decompose computation and create a compositional proof that is within the capacity of an SMT solver. 
% Visual inspection between Cryptol and the NIST standard is simplified through the creation of Cryptol utility functions.
% These utility functions wrap the complex list comprehensions required for looping in Cryptol.
% The results of this work indicates that a modular approach in cryptographic algorithm design greatly simplifies the verification process, as exemplified by SHA-3.

\keywords{Formal Verification \and Cryptography \and \shaThree \and \cryptol \and \saw.}
\end{abstract}

\section{Introduction}\label{sec:introduction}
  
  
The Secure Hashing Algorithm (SHA) family of cryptography is widely used as a means of providing increased security through generating/verifying digital signatures, deriving keys, and generating pseudorandom bits (\href{https://csrc.nist.gov/publications/detail/fips/202/final}{NIST specification}).  
The structure and computations of these algorithms is defined by NIST in mathematical specifications.  
Because these specifications are just figures on a page, any assurance of correctness of an algorithm’s implementation is dependent on a visual review of the code coupled with a small sampling of test vectors.

The reliability of this correctness check is often weakened because of the difficulty of performing that visual review.  
Many implementations of the SHA algorithms go through major rewrites and optimizations which make visual inspection of the code nearly impossible. (Not sure if that’s too strong a claim).
This work bridges that gap between optimized c code, and the NIST standard.

Give a brief history of \shaThree. Introduce briefly SAW and Cryptol.

% Contributions:

% (1) A Cryptol specification for SHA3 that passes test vectors and is readily comparable to the FIPS 202 specification
% (2) A Cpyptol specification for OpenSSL's implementation of SHA3 that passes test vectors and is readily comparable to the the OpenSSL C-code—this specification uses in the Keccak core pre-computed tables and a re-ordered state-vector
% (3) A SAW proof that the FIPS 202 specification is equivalent to the OpenSSL specification
% (4) A SAW proof that the OpenSSL Keccak specification is equivalent to the OpenSSL C-implementation of the same function for __**all possible state vectors**__

% Limitations:

% * Visual inspection and limited test vectors for (1) and (2)
% * Only provide proof certificates for message sizes of X to Y for digest sizes of … for (3)
% * Visual inspection only for anything above Keccak for (3) in OpenSSLs C-code

% @Ben: will you clarify your second point on the visual inspection for 2? I think what you are saying is that it is the FIPS 202 spec with the modified state vector that is what is used in OpenSSL, so there is no value in “visual inspection” in that sense. It which case I would restate (2) as

% (2) A Cpyptol specification based on OpenSSL implementation of SHA3 that uses a re-ordered state-vector and pre-computed tables in Keccak for performance gains that passes test vectors

% To a much lesser extent, I'd say another contribution is that we've shown that when a cryptographical spec is designed in a modular way (as the modern SHA3 is compared to SHA2) that facilitates formal verification---at least if using Cryptol and SAW. I believe we showed this by the fact that the SHA3 proofs complete successful for Yices, Z3, and ABC, whereas with our proofs for SHA2 only Yices completed.

\noindent \textbf{Contributions}:
This paper details the following contributions.
\begin{compactitem}
  \item  A Cryptol specification for \shaThree\ that directly corresponds to the \nist\ \fips\ specification and passes the test vectors provided by the same specification.
  \item A Cryptol specification based on the \openssl\ implementation of \shaThree\ that uses a reordered state vector and precomputed tables in Keccak for performance gains and passes test vectors
  \item A \saw\ proof that the \fips\ specification is equivalent to the OpenSSL specification
  \item A \saw\ proof that the \openssl\ \keccak specification is equivalent to the \openssl\ C-implementation of the same function for all possible state vector input
  \item Early evidence that modular cryptographic specifications such as \shaThree lend themselves to efficient \saw\ verification against C implementations as opposed to flattened specifications such as \shaTwo. 
\end{compactitem}

\begin{comment}
\noindent \textbf{Limitations}
\begin{compactitem}
  \item Visual inspection and limited test vectors for (1) and (2)
  \item Only provide proof certificates for message sizes of X to Y for digest sizes of \textbf{FIXME} for (3)
  \item Visual inspection only for anything above Keccak for (3) in OpenSSL's C-code
\end{compactitem}

\subsection{Related Work}
  Previously, Galois' Inc. worked with Amazon to prove out parts of Amazon's s2n HMAC code to the NIST specification.  
In their work they stubbed out use of the SHA2 algorithm because their HMAC allows for customization of the algorithm type and implementation.

The work by Decker et. al, extended on Galois' work by choosing a single SHA2 implementation to prove out.  
They chose OpenSSL's SHA 256 implementation, as that is a commonly used library (and one that is usable in s2n).  
They were able to prove out the inner hash computations of that implementation using a methodology of compositional proofs to do so.  
They were also able to connect up their proof with the code used by Galois' to stub out the SHA2 algorithm, which implies that the functional equivalence property established with the OpenSSL code applies to Galois' Cryptol implementation as well.  
Their work showed the viability and limitations of creating proofs of functional equivalence for legacy cryptographic implementations.



\subsection{SHA 3}
This work shows that this same methodology can be extended to modern cryptographic algorithms.
The SHA 3 family of algorithms is defined by NIST in the FIPS 202 publication.
It specifies a block hashing algorithm, Keccak, that is used by each of the different SHA and SHAKE algorithms.
In comparison to the SHA 2 family of algorithms, Keccak features significantly more complex and robust computations.
This increase in computation was of interest, because it presses the limits of SAW's capabilities.

\subsection{Results}
This work shows a complete proof of functional equivalence from a verified implementation of the NIST standard to OpenSSL's C code for the Keccak function.
In doing so, it shows SAW's ability to stand up against more difficult computations.
It also provides insights into using Cryptol's functional nature to further simplify the step of visually verifying Cryptol code to the NIST standard.

Because of how NIST defines the standard, the keccak function is used for each digest size of the SHA3 algorithm, with the only significant differences being in the padding/sponging parts of the algorithm, which falls outside this work's scope.
This means that the proof of correctness for this function extends usefulness beyond just the SHA3-256 level which we focused on.
In our repository, we wrote the Cryptol specification to be able to run any of the SHA3 family of algorithms using the same keccak.
A simple extension of this work would be to take that already created and verified Cryptol specification, and prove it up against OpenSSL's slightly modified keccak versions.
\end{comment}

\section{Background}\label{sec:background}
\cryptol\ and \saw\ are used to prove properties of \openssl's implementation of \shaThree\ relative to the \nist\ \fips\ publication \cite{fips202}. 
What the proofs actually say, and an understanding of their limitations, relies on (1) a basic knowledge of how \saw\ works and what \cryptol\ does relative to \saw, and (2) how the \shaThree\ algorithm computes a digest.
This section provides that basic knowledge of the tools and a general overview of the \shaThree\ algorithm.

\subsection{\cryptol\ and \saw}\label{subsec:saw}
\cryptol\ is a domain specific functional language for describing cryptographic primitives, such as algorithms to compute digests, at the bit-level.
Cryptographic primitives are the building blocks for complex cryptographic protocols, such as SSL, that provide a broad range of security guarantees.
\cryptol\ is executable, being able to run test vectors to validate specifications, and more importantly, it compiles directly to \sawcore\ which is the language for formal models in \saw.

\saw\ is a tool to extract formal models from programs with support for C, Java, and \cryptol\ input.
These formal models are expressed in the \sawcore\ language that supports formal reasoning.
As mentioned previously, \cryptol\ compiles directly to \sawcore, but that is not the case for C and Java.
\saw\ uses \emph{symbolic execution} to extract \sawcore\ models from C and Java inputs.

Symbolic execution reasons about all computation paths in the C or Java input.
It must unroll all loops a static number of times to create the model in \sawcore.
As such the number of iterations in any loop must be statically known at compile time or the symbolic execution fails to terminate.
\sawcore\ models from symbolic execution tend to be more complex than those from \cryptol.

\saw's function, among other things, is to prove input to output equivalence between \sawcore\ models.
It accomplishes such proofs with equivalence preserving rewrites on the \sawcore\ and with automated reasoning.
The rewrites simplify the \sawcore\ models as size correlates with complexity in the automated reasoning. 
The rewrites also try to structurally reduce one model to the other thus avoiding automated reasoning altogether.
When rewriting is not sufficient, the equivalence between two models is reduced to a satisfiability problem and dispatched to external backend solvers.

\saw\ supports hierarchical reasoning with \emph{overrides}.
An override replaces one \sawcore\ model with another in the formal analysis.
As an example, consider a Java method that itself calls another Java method as part of its implementation.
Suppose that \saw\ is able to prove that a \cryptol\ specification for the called method and the Java implementation of the same method are equivalent.
If that is possible, then an override is able to replace the called Java method with the \sawcore\ from the \cryptol\ and as a result avoid any symbolic execution of that method when creating the \sawcore\ model of the Java.

As mentioned previously, \sawcore\ from \cryptol\ is less complex than equivalent models created by symbolic execution from C or Java; thereby overrides reduce the overall complexity of the final extracted \sawcore\ models from C or Java.
As such, proving a property about a complex C function or Java method generally requires the use of overrides from \cryptol\ in order for the equivalence proof to be successful, and that is regardless of how that proof is accomplished.
Overrides improve the likelihood of the rewrites showing equivalence and the likelihood of the backend solver being able to show equivalence if the rewrites fail.
Indeed, the use of overrides are key to the \saw\ proof in this paper showing equivalence between the \shaThree\ specification and the \openssl\ C implementation.

\subsection{SHA-3 Overview}\label{subsec:sha3}
Overview the SHA-3 algorithm.
Add enough detail for the proof-outline and results to make sense in the next section.





\section{Proof Outline and Results}\label{sec:proof}
\begin{figure*}[t]
  \centering
  \includegraphics[width=\linewidth]{figs/proof.png}
  
  \caption{Equivalence proof for the \shaThree\ standard and the \openssl\ C implementation.}
  \label{fig:proofStructure}
  
\end{figure*}

The goal of the work presented in this paper is to prove the \openssl\ implementation of \shaThree\ matches the \fips\ publication by \nist.
\figref{fig:proofStructure} illustrates the proof strategy.
The proof begins on the left of the figure with the box labeled \emph{NIST Standard}. 
This box represents the \fips\ standard that defines the \shaThree\ algorithm with english prose and math.
The standard also provides several test inputs with the corresponding expected digests.
The first step of the proof reproduces the \fips\ \shaThree\ description in \cryptol.
That is the box to the right of the \emph{NIST Standard} box labeled \emph{Spec}.

\subsection{Visual Inspection and Test Vectors}

The equivalence between the \fips\ standard and the \cryptol\ specification is argued by visual inspection along with tests over the published test vectors with their expected digests.
\shaThree\ is described algorithmically in the standard making use of universal quantifiers and other looping structures in the definition.
\cryptol\ is functional in that it uses list comprehensions for iteration, and it does not have quantifiers.
Visually certifying a specification of \shaThree\ that uses list comprehensions is less direct.
The list comprehensions can be complex, and at times, they obscure the algorithmic, and mathematic, structure in the published standard.

The syntactic disconnect between the quantification and looping in the published standard and the functional nature of the \cryptol\ specification is bridged with a library that hides list comprehensions with appropriately named functions.
The final specification using the library has a more direct correspondence to the published standard that is more obviously visually certified.
It size is around $200$ \emph{lines of code} (LOC) in \cryptol.

The specification is further validated by tests in \cryptol\ over the published test vectors.
The digests computed from the \cryptol\ specification exactly match those published in the standard on each given input.
The equivalence arrow between the \emph{NIST Standard} box and the \emph{Spec} box represents the visual certification and test vector results.
\secref{sec:fips} details the library using examples from the specification.

\subsection{Memory Layout and Optimized Computation}

The extracted \sawcore\ from the \openssl\ C implementation of \shaThree\ is too big and complex for \saw\ to reason about directly.
Adding to the complexity of the equivalence proof is that the \openssl\ implementation changes the meaning of the state $S$ in the algorithm so that it no longer directly matches the meaning defined in the published standard.
It also changes the computation of the inner \keccak\ functions.
These changes effectively reorder the matrix definition of the state to be more amenable to memory, and optimize the different functions in \keccak\ to be more efficient when operating on the state.
The changes are significant enough to make it extremely difficult to argue manually that the computation implemented in \openssl\ matches that defined in the standard.

The next step of the equivalence proof creates a \cryptol\ specification that matches the state reordering in the \openssl\ implementation, but still follows the bit-level updates in the \fips\ standard in an effort to show an equivalence between it and the \cryptol\ specification for the standard.
This new specification is the box in the right middle of \figref{fig:proofStructure} labeled \emph{OpenSSL} with the \emph{CRY} annotation in the upper right corner and is around 129 LOC. \saw\ is not able to reason symbolically about input and digest sizes to construct a general proof of equivalence between the published standard and the \openssl\ \cryptol\ specification with the state reordering, but it is able to prove equivalence between the two specifications given specific sizes for the input and digest.

A series of \saw\ proofs over a range of input sizes with the digest size fixed at $256$ bits is used to prove the equivalence between the published standard and the \openssl\ \cryptol\ specifications.
The series of proofs varies the input size from zero bits to $1,087$ bits to reflect the $1,088$ bit rate required for a 256 bit digest.
The proofs establish the correctness of the input padding and cover the situation where the padding adds an additional $1,088$ bit block to the input.
The total running time for these proofs is around 60 hours.
As such, the meaning of the equivalence arrow between the \emph{Spec} box and the \emph{OpenSSL} box in the middle of the figure is that the two specifications have equivalent computation for a $256$ bit digest for message sizes from $0$ bits to $1,087$ bits.

\begin{figure*}[t]
  \centering
  \includegraphics[width=\linewidth]{figs/proof2.png}
  
  \caption{Inner Function Contracts}
  \label{fig:proofStructure2}
  
\end{figure*}

\subsection{Overrides and \keccak}

The rest of the equivalence proof between the standard and the \openssl\ implementation centers on the \keccak\ algorithm as shown in the bottom right of \figref{fig:proofStructure}.
The meaning of the equivalence arrow between the \keccak\ definition in the \cryptol\ specification based on the \openssl\ implementation and the actual implemented \keccak\ algorithm in the \openssl\ C implementation is that their computation exactly matches for any input state.
That proof cannot be constructed directly by \saw\ since the extracted \sawcore\ from the C implementation is too big for automated reasoning when considering all twenty-four rounds required by \keccak.
The disconnect is largely a result of the C implementation operating in 64-bit words at a time where the specification is bit level.
The full proof requires overrides.

The proof of equivalence between the specification of \keccak\ and the C implementation of \keccak\ in \openssl\ relies on overrides in \saw\ as shown in \figref{fig:proofStructure2}.
Here, \saw\ proves equivalence between the bit-level \cryptol\ specifications and the word level C implementation for each of the five functions that comprise one round of \keccak\ given some arbitrary input state.
That is the meaning of the left hand side of \figref{fig:proofStructure2}.
These are equivalent on any state $S$.

The implementations for each of the five \keccak\ functions in the C implementation are replaced by the corresponding equivalent \cryptol\ specifications in the extracted model from C.
\saw\ is then able to prove the C implementation of \keccak\ equivalent to the \cryptol\ specification of the \openssl\ implementation of \keccak\ over all twenty-four rounds for any arbitrary input state.
The equivalence is independent of the digest size, message size, or state.
As a minor note, the \openssl\ code for the $\iota$ function was modified in one place to remove an assert-statement safe-guarding an index into a table of constants since the assert-statement is not supported by \saw's symbolic execution engine. The proof takes less than 15 minutes including the proofs for the overrides.

%% TODO: update the figure and text with results from the proof of the padding and the squeeze functions.

\subsection{Results and Summary}

\tableref{table:stats} is a summary of the cost of the verification in running time and LOCs. These are from fairly banal hardware seen in any laptop.
The \cryptol\ specification of the \fips\ standard relies on visual inspection and test vectors for equivalence having to look at around 200 LOC.
The equivalence between the \cryptol\ specification of the standard and the \cryptol\ specification of the \openssl specification that reorders the state array is limited to the 256 bit digest on input sizes from $0$ bits to $1,087$ bits and takes around 60 hours of computation time.
That is a proof that considers all of \algoref{alg:sha3}.
The equivalence between the \openssl\ \cryptol\ specification that reorders the state array and the actual C implementation is limited to the \keccak\ algorithm only.
That is a proof that only considers \lineref{line:keccak} in \algoref{alg:sha3} but holds for any state and is independent of the digest size and rate.
The cost of proving the equivalence for each of the overrides is trivial with the exception of $\iota$.
$\iota$ is more costly because it uses a lookup table in the C code so every entry in the table had to be proved out (see \secref{sec:openssl}).
The final cost of proving the equivalence in \keccak\ with the overrides is less than 15 minutes with the \texttt{yices} solver.

\begin{table}[t]
  \caption{Proof runtimes (RT) and cumulative code sizes (CS)}\label{statsTable}
  \setlength{\tabcolsep}{6.5pt}
  \begin{tabular}{|l|l|l|l|l|}
  \hline
  \textbf{Function} & \textbf{yices RT} & \textbf{abc RT} & \textbf{Cryptol CS} & \textbf{C CS} \\
  \hline
  Pi & 	0.8951595s & 0.4171096s & 7 lines & 35 lines \\
  Rho & 4.5090218s & 5.879199s & 19 lines & 41 lines \\
  Theta & 25.9416942s & 14.8692876s & 17 lines & 55 lines \\
  Chi & 5.0592874s & 1.3271656s & 7 lines & 18 lines \\
  Iota & 11.0476514s & 612.4968917s & 37 lines & 31 lines \\
  Keccak (With OV's) & 745.205611s &	2910.5493025s & 93 lines & 192 lines \\
  \hline
  \end{tabular}
  \end{table}


\section{\fips\ Specification in \cryptol}\label{sec:fips}
This section illustrates with a few examples how the \fips\ specification is captured in \cryptol. 
\cryptol's list comprehensions do not directly align with the quantification and iteration in the specification. 
This misalignment makes the \cryptol\ specification more difficult to read and to manually argue its equivalence relative to the published specification.
The two specifications are better aligned through a library of two methods in \cryptol\ that hide the list comprehensions so that the \cryptol\ specification reads more like the \fips\ specification. The two library methods are the \emph{for-method} and the \emph{while-method} discussed in this section.

\subsection{The for-method}

\newsavebox{\fipsPi}
\begin{lrbox}{\fipsPi}
  \begin{lstlisting}[basewidth = {.5em},basicstyle={\small}]
    1. For all triples (x,y,z) such that 
           0 <= x < 5, 0 <= y < 5, and 0 <= z < w,
       let A'[x, y, z] = A[(x + 3y) mod 5, x, z].
    2. Return A'.
  \end{lstlisting}
\end{lrbox}

\newsavebox{\PiCry}
\begin{lrbox}{\PiCry}
  \begin{lstlisting}[language=Cryptol, numbers=left, escapeinside={;}{;}]
    type STATE_ARR = [5][5][64];\label{line:pistate};
    pi : STATE_ARR -> STATE_ARR
    pi a = [  [  [a @x @((x + 3*y) % 5) @z;\label{line:pi};
                 | z <- [0..63]];\label{line:list};
              | y <- [0..4]] 
           | x <- [0..4]]
  \end{lstlisting}
\end{lrbox}

\newsavebox{\PiCryLib}
\begin{lrbox}{\PiCryLib}
  \begin{lstlisting}[language=Cryptol]
    pi a = for [0..4] (&x ->
             for [0..4] (&y -> 
               for [0..63] (&z -> 
                 a @x @((x + 3*y) % 5) @z)))
  \end{lstlisting}
\end{lrbox}

\newsavebox{\formethod}
\begin{lrbox}{\formethod}
  \begin{lstlisting}[language=Cryptol]
    for : {n, a, b} [n]a -> (a -> b) -> [n]b
    for vals f = [f i | i <- vals]
  \end{lstlisting}
\end{lrbox}

\begin{figure}[t]
  \begin{center}
    \begin{tabular}{l}
      \usebox{\fipsPi} \\ \\
      \multicolumn{1}{c}{(a)} \\ \\
      \usebox{\PiCry} \\ \\
      \multicolumn{1}{c}{(b)} \\ \\
      \usebox{\formethod} \\ \\
      \multicolumn{1}{c}{(c)} \\ \\ 
      \usebox{\PiCryLib} \\ \\
      \multicolumn{1}{c}{(d)}
    \end{tabular}
  \end{center}
  \caption{The $\pi$ function. (a) The \fips\ specification. (b) The \cryptol\ with comprehensions. (c) The for-method definition. (d) The \cryptol\ with the for-method.}
  \label{fig:pi}
\end{figure}

A common idiom in the \fips\ publication is using quantification over a finite domain defined by linear constraints to define input for some computation.
The definition of the $\pi$ function for \keccak\ shown in \figref{fig:pi}(a) is one such example.
Here the quantification is used to define the set of indices that are part of the $\pi$ computation that transforms the state array. The intent is that of a \emph{parallel-for} where order does not matter since the new state array $A^\prime$ only depends on the old state array $A$.

The \cryptol\ definition using list comprehensions is given in \figref{fig:pi}(b).
\lineref{line:pistate} defines the state as a $5 \times 5$ array of 64-bit words.
\lineref{line:pi} starts the definition of $\pi$. 
Here the comprehensions are nested with the list indices appearing in reverse order to follow the nesting.
\lineref{line:list} gives the domain for $z$ as a list with values from 0 to 63.
The domains for $x$ and $y$ are given similarly.
The domains for $x$, $y$, and $z$ all match the domains defined in the quantification in \figref{fig:pi}(a).
Although the list comprehensions are not unreadable, the \cryptol\ can be improved by creating a for-method to hide some of the details.

\figref{fig:pi}(c) is the \cryptol\ definition of the for-method.
It takes two arguments: a list, \emph{vals}, for the comprehension, and a function, $f$, to apply to each element of the list.
Its definition is as expected: it creates a new list that is the result of applying $f$ to each element in \emph{vals}.

\figref{fig:pi}(d) is the rewritten \cryptol\ using the for-method.
It still requires the indices lists for the list comprehensions, but it has a more obvious correspondence to the original definition in \figref{fig:pi}(a).
The for-method reorders and structures the arguments in a very elegant way.
It hides the list comprehensions to simplify the task of visual inspection for equivalence.

\subsection{The while-method}

\newsavebox{\fipsRc}
\begin{lrbox}{\fipsRc}
  \begin{lstlisting}[basewidth = {.5em},basicstyle={\small}]
    1. If t mod 255 = 0, return 1.
    2. Let R = 10000000.
    3. For i from 1 to t mod 255, let:
      a. R = 0 || R;
      b. R[0] = R[0] ^ R[8];
      c. R[4] = R[4] ^ R[8];
      d. R[5] = R[5] ^ R[8];
      e. R[6] = R[6] ^ R[8];
      f. R = Trunc8[R].
    4. Return R[0].
  \end{lstlisting}
\end{lrbox}

\newsavebox{\RcCry}
\begin{lrbox}{\RcCry}
  \begin{lstlisting}[language=Cryptol, numbers=left, escapeinside={;}{;}]
    rc : [64] -> Bit
    rc t =  if (t % 255) == 0 then 1
            else rs !0 @0 where;\label{line:return};
              rs = [0b10000000] #;\label{line:init};
                   [if i <= t % 255 then;\label{line:compute};
                      ((nextR r) where r = [0] # (rs @(i - 1)))
                    else rs @(i - 1);\label{line:stutter};
                   | i <- [1..254]];\label{line:alliters};
  \end{lstlisting}
\end{lrbox}

\newsavebox{\whilemethod}
\begin{lrbox}{\whilemethod}
  \begin{lstlisting}[language=Cryptol]
    while : {a} a -> (a -> Bit) -> (a -> a) -> a
    while state cond f = 
      if (cond state) then (while (f state) cond f)	
      else state
  \end{lstlisting}
\end{lrbox}

\newsavebox{\RcCryLib}
\begin{lrbox}{\RcCryLib}
  \begin{lstlisting}[language=Cryptol, numbers=left, escapeinside={;}{;}]
    rc : [64] -> Bit
    rc t =  if (t % 255) == 0 then 1 
            else (while {i = 1, R = 0b10000000};\label{line:state};
                        (&state -> state.i <= t % 255);\label{line:cond};
                        (&state -> {
                          i = state.i + 1,;\label{line:inc};
                          R = (nextR r) where r = [0] # state.R})
                 ).R @0;\label{line:libreturn};
  \end{lstlisting}
\end{lrbox}

\begin{figure}[t]
  \begin{center}
    \begin{tabular}{l}
      \usebox{\fipsRc}
    \end{tabular}
  \end{center}
  \caption{The \fips\ specification for \emph{rc}.}
  \label{fig:rc}
\end{figure}

Another common idiom in the \fips\ standard is to iteratively transform some state a fixed number of times to arrive at a final state.
The definition of the \emph{round constant}, \emph{rc}, function in \figref{fig:rc} is one such example.
The \emph{rc} function is a support function for the $\iota$ function in \keccak\ and is used to build a mask.

The \emph{rc} function checks the value of the input integer $t$, and returns the bit $0$ if it is a multiple of $255$.
If not, then Line 2 defines $R$ as the 8-bit value $128$, and Line 3 iteratively transforms $R$ a fixed number of times: $t\ \mathbf{mod}\ 255$.
The transform is defined on Line 3a through Line 3f.
Line 3a prepends a $0$ bit to $R$ so that it is now a 9-bit value. The array indexing refers to bit positions from left to right so $R[0]$ is the prepended left-most bit and $R[8]$ is the right-most bit. Each transform on $R$ prepends the 0 (Line 3a), does an exclusive-or with specific bit positions (Line 3b through Line 3e), and then takes the left-most 8 bits for the next round effectively dropping the right-most bit (Line 3f).
It returns the $R[0]$ bit at the end.

The \cryptol\ equivalent for \emph{rc} in \figref{fig:rccry}(a) is considerably more complex with the list comprehensions.
Returning $1$ when the input is a multiple of $255$ is straightforward.
It is when that is not the case that it is less obvious.
The list comprehension builds a list of $255$ 8-bit values.
\lineref{line:return} returns the bit at index 0 ($@0$) from the last 8-bit value in the array ($!0$).
The computation of the list of 8-bit values follows.

\begin{figure}[t]
  \begin{center}
    \begin{tabular}{l}
      \usebox{\RcCry} \\ \\
      \multicolumn{1}{c}{(a)} \\ \\
      \usebox{\whilemethod} \\ \\
      \multicolumn{1}{c}{(b)} \\ \\ 
      \usebox{\RcCryLib} \\ \\
      \multicolumn{1}{c}{(c)}
    \end{tabular}
  \end{center}
  \caption{The \emph{rc} function in \cryptol, where \emph{nextR} elides steps 3b to 3f of \figref{fig:rc}. (a) The \cryptol\ with list comprehensions. (b) The recursive while-method definition. (c) The \cryptol\ with the while-method.}
  \label{fig:rccry}
\end{figure}

\lineref{line:init} is the first entry in the list and it is the initial value of $R$.
The following entries depend on the value of the input $t$ and the value of $i$ that comes from the list used in the list comprehension on \lineref{line:alliters}.
\lineref{line:compute} checks if $i$ is within the number of iterations, and if it is, then the new entry in the list is computed from the previous entry as defined by the transform in \figref{fig:rc}.
\lineref{line:stutter} is the case for when $i$ is not within the number of iterations.
Here the previous value in the list is copied to stutter the last value of $R$. 
The \cryptol\ specification creates all $255$ entries in the list regardless of the input value $t$ except for the case when $t$ is a multiple of 255.
As before, \figref{fig:rccry}(a) is not altogether unreadable, but it can be improved considerably.

\figref{fig:rccry}(b) is a recursive definition of a \emph{while-loop}.
It takes as input a state, a predicate on the state, \emph{cond}, that is true when it should loop and false otherwise, and a function, $f$, that computes a new state from the old state on each iteration.
The definition makes recursive calls to the while-method until the state predicate is false at which point it returns the current value of \emph{state}.

\figref{fig:rccry}(c) is the definition of \emph{rc} using the while-method in \figref{fig:rccry}(b).
\lineref{line:state} defines the state for the while loop to be both an integer $i$ for a counter and the 8-bit value $R$.
\lineref{line:cond} is the looping condition defined on the state.
The body of the loop not only computes the new value of $R$ but \lineref{line:inc} increments $i$ to track the loop iteration.
The return value is given on \lineref{line:libreturn} as the $0^\mathrm{th}$ bit from the last value of $R$.
Here, $R$ is computed only the number of times required by $t$.
This change differs significantly from the definition without the while-method.
As with the for-method, the while-method reduces the gap between how things are defined in \fips\ and how they are defined in \cryptol.


\section{\openssl\ Differences}\label{sec:openssl}
This section details the key differences between the \openssl\ implementation of \shaThree\ and the \fips\ specification that prevented a direct proof of equivalence between the \cryptol\ and C code.
These differences are expressed in a second \cryptol\ model derived from the \fips\ model discussed in the previous section.
\saw\ proves these two models equivalent for the 256-bit digest size as discussed in \secref{sec:proof}.
This equivalence considers the whole of the sponge construction algorithm proving that the digest is the same from each of them.
That said, and as a reminder, the equivalence between the \keccak\ description in \cryptol\ and the C implementation in \openssl\ holds for any input message and any digest size.
That proof uses the \cryptol\ definition for the model discussed in this section that includes all the differences seen in \openssl, and the overrides used in that proof also come from the \cryptol\ discussed in this section.

\subsection{State Array Structure and Computation}

The first set of differences in the \openssl\ \shaThree\ implementation is in the structure of the state array and how it operates on that state array.
The difference in structure is an artifact of how C maps arrays to memory.
The \fips\ structures the state array as a $5 \times 5$ grid of 64-bit words with each 64-bit word being a \emph{lane}.
It assumes the layout of the data follows a normal cartesian three dimensional coordinate system with $x$ being the horizontal axis, $y$ being the vertical axis, and $z$ being the depth on a lane to access an individual bit.
For a state $A$, $A[x,y,z]$ accesses the $z^\mathrm{th}$ bit from the 64-bit word on the $x^\mathrm{th}$ column and the $y^\mathrm{th}$ row.

\openssl\ declares the state as follows: \texttt{uint64\textunderscore t A[5][5]}.
The C standard stores multidimensional arrays in contiguous memory in \emph{row-major} order meaning that each row of five 64-bit values appear consecutively in memory.
Indexing the multidimensional array follows the standard mathematical definition for indexing matrices: $A[x,y]$ is the element at the $x^\mathrm{th}$ row and $y^\mathrm{th}$ column.
This meaning is just opposite of that in the standard.

Adding to the complexity is that the C standard does not provide array indexing to get a bit from a value.
For the $A[x,y,z]$ example, there is no array bracket notation to get the $z^\mathrm{th}$ bit in a 64-bit word, so notation in the \fips\ standard such as that seen in the definition of the $rc$ function in \figref{fig:rc}(a) has no direct analogue in C.
The consequence is that the \openssl\ implementation does everything at the level of the lanes, operating on each lane as a 64-bit entity, and it never refers to an individual bit in a lane.
Finally, the bit ordering in the lanes in the \fips\ standard is just opposite the ordering in C meaning that the direction of shifting in the standard is opposite the direction used in the C implementation.

The \cryptol\ for the \fips\ model is rewritten to reflect these memory layout and computation differences.
The $x$ and $y$ indexing is swapped.
All the inner \keccak\ functions are modified to operate on lanes in their entirety.
And the the shifts are reversed.
For reference, \figref{fig:piopenssl} is the rewritten $\pi$ function in \cryptol\ than matches \openssl's implementation (compare to \figref{fig:pi}(d)).
Operating on lanes in faster and more efficient, and that change is reflected in all of the inner \keccak\ functions.

\newsavebox{\PiOpenSSL}
\begin{lrbox}{\PiOpenSSL}
  \begin{lstlisting}[language=Cryptol]
    pi a = for [0..4] (&y ->
             for [0..4] (&x -> 
                 a @x @((x + 3*y) % 5))
  \end{lstlisting}
\end{lrbox}

\begin{figure}
  \begin{center}
    \usebox{\PiOpenSSL}
  \end{center}
  \caption{The \openssl\ implementation of $\pi$ that reverses indexes and operates on lanes.}
  \label{fig:piopenssl}
\end{figure}

\subsection{Constant Lookup Tables}

\newsavebox{\ciota}
\begin{lrbox}{\ciota}
  \begin{lstlisting}
    void Iota(uint64_t A[5][5], size_t i) {
      //assert(i < (sizeof(iotas) / sizeof(iotas[0])));
      if (i < (sizeof(iotas) / sizeof(iotas[0]))) {
        A[0][0] ^= iotas[i];
      }
    }
  \end{lstlisting}    
\end{lrbox}

% iota : STATE_ARR -> [64] -> STATE_ARR
% iota a i = for [0..4] (\y ->
%               for [0..4] (\x ->
%                   for [0..63] (\z ->
%                       if ((x == 0) && (y == 0))
%                           then (a @0 @0 @z) ^ (LISTIOTAS @i @z)
%                       else a @y @x @z)))

\newsavebox{\cryciota}
\begin{lrbox}{\cryciota}
  \begin{lstlisting}[language=Cryptol]
    LISTIOTAS = [reverse (
      (while {RC = 0:[W], j = 0} 
              (\state -> state.j <= `L) 
              (\state -> { 
                RC = for [0..63] 
                  (\z -> if z == index then rc (state.j + 7*i)
                          else state.RC @z)
                    where index = ((1:[8]) << state.j) - 1,
                                    j = state.j + 1})
      ).RC) | i <- LISTROUNDS:[_][64]]

    LISTIOTAS = [
      0x0000000000000001, 0x0000000000008082, 0x800000000000808a,
      0x8000000080008000, 0x000000000000808b, 0x0000000080000001,
      0x8000000080008081, 0x8000000000008009, 0x000000000000008a,
      0x0000000000000088, 0x0000000080008009, 0x000000008000000a,
      0x000000008000808b, 0x800000000000008b, 0x8000000000008089,
      0x8000000000008003, 0x8000000000008002, 0x8000000000000080,
      0x000000000000800a, 0x800000008000000a, 0x8000000080008081,
      0x8000000000008080, 0x0000000080000001, 0x8000000080008008]
        :[_][64]
  \end{lstlisting}    
\end{lrbox}

\begin{figure}[t]
  \begin{center}
    \begin{tabular}{l}
      \usebox{\cryciota} \\ \\
    \end{tabular}
  \end{center}
  \caption{Optimized $\iota$ function. (a) \openssl's implementation. (b) Matching \cryptol\ implementation.}
  \label{fig:iota}
\end{figure}

The second difference is in hardcoding lookup Tables for constants rather than computing the constants along the way.
The \fips\ \cryptol\ specification computes the lookup table once for the rounding constants, \emph{rc}, for $\iota$, and it computes the rotate constants for $\rho$ on the fly.
The \openssl\ specification hardcodes these constants in tables without the provenance of the computation.
\figref{fig:iota} shows the difference in the two specifications for the rounding constants with the code for the computed table coming directly from \fips.

\section{Related Work}\label{sec:related}
Cryptol has been used to formally verify different implementations of the same cryptographic algorithm in the past.
As a demonstration of Cryptol's applications, Lewis et al (CRYPTOL: HIGH ASSURANCE, RETARGETABLE CRYPTO
DEVELOPMENT AND VALIDATION) provided an portions of the Advanced Encryption Standard implemented in the language.
The language has also been used to verify equivalence between Skein Hash algorithm implementations (Hardware/Software Co-verification of Cryptographic Algorithms using Cryptol).
Cryptol is not limited to hardware, as Browning and Weaver (Designing Tunable, Verifiable Cryptographic Hardware Using Cryptol) have shown that the language can be tuned to rovide an abstraction of hardware.
They later provided a proof of equivalence between the FIPS AES specificiation and various optimized Cryptol implementations.
For certain tables, they initially demonstrated equivalence through mathematical principles.
These tables were also proven equivalent to less time-intensive statically defined tables used for the rest of their proofs.
This was mirrored in our own cryptol files as the OpenSSL implementation was highly optimized with similar static values.
The language was also used to verify equivalence between the specification and hardware of the Verilog RTL's pseudorandom number generator (Formal Verification and Analysis of a Pseudo Random Number Generator).

Decker et al's verification of OpenSSL's SHA2-256 implementation served as a direct predecessor to this work (Towards Verifying SHA256 in OpenSSL with the Software Analysis Workbench).
While SHA2 is also part of NIST's Secure Hash Algorithms, its internal structure differs vastly from the more complex and resistant SHA3.
The SHA2 proof, in turn, expanded the methodology by which Amazon's s2n HMAC was verified (Verifying s2n HMAC with SAW).
While SHA3 is not implemented in s2n, the HMAC verificatiprovided a on utilized both a high-level and low-level cryptol specification in a similar manner to this work.
Both works compared equivalent C and Cryptol specification through SAW.

Other tools have been used on a wide variety of cryptographic algorithms.
STP(A Decision Procedure for Bit-Vectors and Arrays), a SAT-based procedure, has been used to verify implementation of AES, DES, SHA-1, and other block cipher functions (Automatic Formal Verification of Block Cipher Implementations).
Isabelle(https://isabelle.in.tum.de/) has aided in formal verification of networking protocols(Formal Verification of Secure Forwarding Protocols).
The Coq(https://coq.inria.fr/) Proof assistant was used to prove equivalence between the OpenSSL, mbedTLS, and NIST implementations of HMAC-DRBG (Verified Correctness and Security of mbedTLS HMAC-DRBG).
Using Jasmine(https://github.com/jasmin-lang/jasmin), Barthe et al showed that implementations of the ChaCha20 and Poly1305 were equivalent to their specifications.
Vale has been used to verify the SHA-256 and AES implementations (Vale: Verifying High-Performance Cryptographic Assembly Code).
More close to home, the VeriHash framework demonstrated a bug in the RHash (https://github.com/rhash/RHash) implementation of SHA3-256 (Verification of Implementations of Cryptographic Hash Functions).
We discovered no bugs in OpenSSL's implementation.


\section{Conclusion}\label{sec:conclusion}
Add conclusions and future work.

%
%
%
%
% ---- Bibliography ----
%
% BibTeX users should specify bibliography style 'splncs04'.
% References will then be sorted and formatted in the correct style.
%
\bibliographystyle{splncs04}
\bibliography{paper}
%
\end{document}
