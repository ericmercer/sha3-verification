Almeida et al give a formal specification of \shaThree\ in Jasmin \cite{10.1145/3319535.3363211}.
They then prove that \shaThree\ is resistant to collisions, correctly implemented by Jasmin in vectorized x86, and that implementation is resistant to side-channel attacks.
The proofs are mechanized and carried out by the EasyCrypt proof assistant \cite{easycrypt}.
The vectorized implementation is performant and efficient.

EverCrypt from Protzenko et al is a library of verified cryptographic primitives including several state of art algorithms for computing digests \cite{9152808}.
Here the algorithms are specified in F* \cite{fstar}.
Automated proofs of properties of the crypto-primitives are done with a weakest-precondition calculus.
As subset of F* is compilable to C the library is compiled to different target backends.
The resulting library is performant and efficient.

The work is this paper differs from the these related efforts in that this work does not prove properties of \shaThree\ itself but rather proves an existing implementation of \shaThree\ is correct relative to the published standard.
Here the verification effort is to prove \openssl\ correct in its implementation of \shaThree.
\openssl\ being a widely deployed implementation written for performance not thinking of verification makes it an interesting target for formal verification and an interesting case study for \saw.

\cryptol\ has been used to formally verify several cryptographic algorithms in the past.
As a demonstration of \cryptol, Lewis \emph{et al} provided portions of the \emph{Advanced Encryption Standard} (AES) implemented in the language \cite{crypt-hi}.
The language has also been used to verify equivalence between Skein Hash algorithm implementations \cite{hard-soft}.
\cryptol\ is not limited to software, as Browning and Weaver \cite{design-verif} have shown that the language can be tuned to provide an abstraction of hardware.
They later provided a proof of equivalence between the FIPS AES specification and various optimized \cryptol\ implementations.
For certain tables, they initially demonstrated equivalence through mathematical principles.
These tables were also proven equivalent to less time-intensive statically defined tables used for the rest of their proofs.
As the OpenSSL implementation was highly optimized with similar static values, this paper mirrored the method Browning and Weaver outlined.
The language was also used to verify equivalence between the specification and hardware of the Verilog RTL's pseudorandom number generator \cite{pseudorandom}.

Decker et al's verification of OpenSSL's \shaTwo\ 256 bit digest implementation served as a direct predecessor to this work \cite{nfm-us}.
While \shaTwo\ is also part of the \nist\ Secure Hash Algorithms, its internal structure differs vastly from the more complex and resistant \shaThree.
The \shaTwo\ proof, in turn, expanded the methodology by which Amazon's s2n HMAC was verified \cite{chudnov,s2n-blog}.
While \shaThree\ is not implemented in s2n, the HMAC verification provided utilized both a high-level and low-level \cryptol\ specification in a similar manner to this work.
Both works compared equivalent C and \cryptol\ specification through \saw.
Decker's work is inspired by the \openssl\ verification of HMAC from Beringer \cite{beringer}.

Other tools have been used on a wide variety of cryptographic algorithms.
STP \cite{stp}, a SAT-based procedure, has been used to verify implementation of AES, DES, \shaOne, and other block cipher functions \cite{auto}.
Isabelle has aided in formal verification of networking protocols \cite{isabelle,formal-protocols}.
The Coq proof assistant was used to prove equivalence between the OpenSSL, mbedTLS, and NIST implementations of HMAC-DRBG \cite{coq,verified-correctness}.
Using Jasmin \cite{jasmin}, Barthe \emph{et al} showed that implementations of the ChaCha20 and Poly1305 were equivalent to their specifications \cite{hi-assurance}.
Vale has been used to verify the \shaTwo\ 256 bit digest and AES implementations \cite{vale}.
Also of interest, the VeriHash framework demonstrated a bug in the RHash implementation of \shaThree\ with the 256 bit digest \cite{rhash,hash-functions}.
