Cryptol has been used to formally verify different implementations of the same cryptographic algorithm in the past.
As a demonstration of Cryptol's applications, Lewis et al (CRYPTOL: HIGH ASSURANCE, RETARGETABLE CRYPTO
DEVELOPMENT AND VALIDATION) provided an portions of the Advanced Encryption Standard implemented in the language.
The language has also been used to verify equivalence between Skein Hash algorithm implementations (Hardware/Software Co-verification of Cryptographic Algorithms using Cryptol).
Cryptol is not limited to hardware, as Browning and Weaver (Designing Tunable, Verifiable Cryptographic Hardware Using Cryptol) have shown that the language can be tuned to rovide an abstraction of hardware.
They later provided a proof of equivalence between the FIPS AES specificiation and various optimized Cryptol implementations.
For certain tables, they initially demonstrated equivalence through mathematical principles.
These tables were also proven equivalent to less time-intensive statically defined tables used for the rest of their proofs.
This was mirrored in our own cryptol files as the OpenSSL implementation was highly optimized with similar static values.
The language was also used to verify equivalence between the specification and hardware of the Verilog RTL's pseudorandom number generator (Formal Verification and Analysis of a Pseudo Random Number Generator).

Decker et al's verification of OpenSSL's SHA2-256 implementation served as a direct predecessor to this work (Towards Verifying SHA256 in OpenSSL with the Software Analysis Workbench).
While SHA2 is also part of NIST's Secure Hash Algorithms, its internal structure differs vastly from the more complex and resistant SHA3.
The SHA2 proof, in turn, expanded the methodology by which Amazon's s2n HMAC was verified (Verifying s2n HMAC with SAW).
While SHA3 is not implemented in s2n, the HMAC verificatiprovided a on utilized both a high-level and low-level cryptol specification in a similar manner to this work.
Both works compared equivalent C and Cryptol specification through SAW.

Other tools have been used on a wide variety of cryptographic algorithms.
STP(A Decision Procedure for Bit-Vectors and Arrays), a SAT-based procedure, has been used to verify implementation of AES, DES, SHA-1, and other block cipher functions (Automatic Formal Verification of Block Cipher Implementations).
Isabelle(https://isabelle.in.tum.de/) has aided in formal verification of networking protocols(Formal Verification of Secure Forwarding Protocols).
The Coq(https://coq.inria.fr/) Proof assistant was used to prove equivalence between the OpenSSL, mbedTLS, and NIST implementations of HMAC-DRBG (Verified Correctness and Security of mbedTLS HMAC-DRBG).
Using Jasmine(https://github.com/jasmin-lang/jasmin), Barthe et al showed that implementations of the ChaCha20 and Poly1305 were equivalent to their specifications.
Vale has been used to verify the SHA-256 and AES implementations (Vale: Verifying High-Performance Cryptographic Assembly Code).
More close to home, the VeriHash framework demonstrated a bug in the RHash (https://github.com/rhash/RHash) implementation of SHA3-256 (Verification of Implementations of Cryptographic Hash Functions).
We discovered no bugs in OpenSSL's implementation.
