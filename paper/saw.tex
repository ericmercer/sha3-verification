\cryptol\ is a domain specific language for describing cryptographic primitives, such as algorithms to compute digests, at the bit-level.
Cryptographic primitives are the building blocks for complex cryptographic protocols, such as SSL, that provide a broad range of security guarantees.
\cryptol\ is executable, being able to run test vectors to validate specifications, and more importantly, it compiles directly to \saw\ core terms for formal reasoning.

\saw\ is a tool to extract formal models from programs with support for C, Java, and \cryptol\ input.
These formal models are referred to as core terms in \saw.
\saw's function, among other things, is to prove input to output equivalence between core terms.
It accomplishes such proofs with equivalence preserving rewrites on the core terms and symbolic execution.
The rewrites simplify the terms for symbolic execution as core term size correlates with complexity. 
The rewrites also try to structurally reduce one term to the other thus avoiding symbolic execution altogether.
Symbolic execution reasons about all computation paths in the core terms and proves equivalence in computation over all inputs.
Symbolic execution must unroll loops in any core term a static number of times to accomplish the equivalence proof.
As such the number of iterations in any loop must be statically known at compile time.

\saw\ supports hierarchical reasoning with \emph{overrides}.
An override replaces one core term with another in the formal analysis.
As an example, consider a Java method that itself calls another Java method as part of its implementation.
Suppose that \saw\ is able to prove that a \cryptol\ specification for the called method and the Java implementation of the same method are equivalent.
If that is possible, then an override is able to replace the called Java method with the core term from the \cryptol\ in the model extraction from Java.
Core terms from \cryptol\ are smaller in size and simpler in structure than equivalent terms extracted from C or Java thereby reducing the size and complexity of the extracted core term in \saw.
As such, proving a property about a complex C function or Java method generally requires the use of overrides from \cryptol\ in order for symbolic execution to be successful.
Indeed, the use of overrides are key to the \saw\ proof in this paper showing equivalence between the \shaThree\ specification and the \openssl\ C implementation.
