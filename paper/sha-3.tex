\begin{algorithm}[t]
  \caption{\shaThree\ sponge construction with \keccak}\label{alg:sha3}
  \begin{algorithmic}[1]
    \Require $n \geq 0$ and $(d,r) \in \{(224, 1152)\ (256, 1088)\ (384, 832)\ (512, 576)\}$
    \Ensure $D =$ \shaThree(N,d,r)
    \State $p_0, \ldots, p_m = \mathtt{pad}(N, r)$\label{line:pad}
    \State $A = 0_0, \ldots, 0_{1599}$\label{line:init}
    \For{$p_i = p_0, \dots, p_m$}
      \State $A = A\ \mathbf{xor}\ \mathtt{zeroExtend}(p_i, a-r)$\label{line:absorb}
      \State $A =$ \keccak$(A)$\label{line:keccak}
    \EndFor
    \State $D = \mathtt{squeeze(A,d)}$\label{line:squeeze}
  \end{algorithmic}
\end{algorithm}

\shaThree\ uses \emph{sponge construction} to generate hashes as shown in \algoref{alg:sha3} (see \nist\ \fips\ \cite{fips202}).
Sponge construction \emph{absorbs} data into an internal state $A$ at a rate of $r$ bits per cycle (\lineref{line:absorb} and \lineref{line:keccak}), transforming the state on each cycle, and once all the data is absorbed, then it \emph{squeezes} out a final $d$-bit digest (\lineref{line:squeeze}).
The transformation function, \keccak-f[1600], simply \keccak\ in the rest of this presentation, defines the size of the state $A$ to be $a = 1600$ bits, which is initialized to 0 (\lineref{line:init}).

The state itself consists of $r$ bits that are written or read plus a capacity of $c = s - r$ bits that are untouched by input or output.
The \nist\ standard defines pairs of allowed hash sizes and rates.
These are defined in the requires statement in \algoref{alg:sha3}. 
The input $N$ is padded such that its final size is a multiple of $r$ bits, and it is partitioned into $r$-bit size blocks (\lineref{line:pad}) for the absorption phase.
The padding is defined by the regular expression $10^*1$.
Padding is always added to $N$ even if adding it creates an extra block.

The \keccak\ transformation function takes place over twenty-four rounds with each round consisting of five steps: $\theta$, $\rho$, $\pi$, $\chi$, and $\iota$.
The state $A$ itself is viewed as a $5 \times 5$ array of $64$-bit words (e.g. $5 * 5 * 64 = 1600$) in each of the steps.
$\theta$ computes a parity over the columns in the state. $\rho$ is a bitwise rotate over the state.
$\pi$ is a permutation on the state. $\chi$ is a non-linear operation that combines along the rows in the state.
And $\iota$ mixes in twenty-four different constants, one for each iteration, into portions of the state.
